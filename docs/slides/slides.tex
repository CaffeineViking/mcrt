\documentclass{beamer}
\usepackage[utf8]{inputenc}
\usepackage{algorithmic}
\usepackage{algorithm}
\usepackage{amsfonts}
\usepackage{amssymb}
\usepackage{courier}
\usepackage{graphicx}
\usepackage{listings}
\usepackage{mathtools}
\usepackage[font={small}, labelfont={color=black}]{caption}
\usetheme{Rochester}

\usefonttheme[onlymath]{serif}
\beamertemplatenavigationsymbolsempty
\title{\LARGE{Global Illumination Using Photon Maps}\\
       \large{\emph{A summary of the paper by Henrik W. Jensen~\cite{jensen1996global}}}}
\author{\vspace{2ex}\\\textbf{Martin Estegren} \;\;\,\,
        \texttt{<mares480@student.liu.se>} \\
        \textbf{Erik S. V. Jansson}\;
        \texttt{<erija578@student.liu.se>} \\~\\
        {Advanced Global Illumination and Rendering}\\
        {at ITN Linköping University (LiTH), Sweden}}

\lstset{
    escapeinside={<@}{@>},
    basicstyle=\tiny\ttfamily,
    breakatwhitespace = false,
    breaklines = true,
    captionpos = b,
    keepspaces = true,
    language = C++,
    showspaces = false,
    showstringspaces = false,
    frame = tb,
    aboveskip = 10pt,
    belowskip = 10pt,
    numbers = left,
    numbersep = 3pt
}

\setbeamertemplate{footline}[frame number]
\begin{document}
    \frame{\titlepage}

    \frame{\frametitle{Global Illumination}\framesubtitle{Rendering Equation}
        \[L_o(\vec{x}, \hat{\omega}_o) = L_e(\vec{x}, \hat{\omega}_o) +
                                         \overbrace{\int_\Omega L_i(\vec{x}, \hat{\omega}_i)
                                         f_r(\vec{x}, \hat{\omega}_i, \hat{\omega}_o)
                                         (\hat{n}_{x} \cdot \hat{\omega}_i) \, d\hat{\omega}_i}
                                         ^{\text{\emph{reflected radiances} from } \Omega \text{ into } \vec{x} \text{ toward } \hat{\omega}_o}\]
        \begin{itemize}
            \item \(L_o(\vec{x}, \hat{\omega}_o)\): total \emph{outgoing radiance} at point \(\vec{x}\) towards a \(\hat{\omega}_o\).
            \item \(L_e(\vec{x}, \hat{\omega}_o)\): \emph{emitted radiance} contribution from \(\vec{x}\) toward \(\hat{\omega}_o\).
            \item \(\Omega\): hemisphere around the point \(\vec{x}\) with normal \(\hat{n}_x\) of \(d \hat{\omega}_i\)'s.
            \item \(L_i(\vec{x}, \hat{\omega}_i)\):\ \emph{incoming radiance} contributions fr.\ \(\hat{\omega}_i\) towards \(\vec{x}\).
            \item \(f_r(\vec{x}, \hat{\omega}_i, \hat{\omega}_o)\): surface \emph{reflectance properties} at \(\vec{x}\), an BRDF.
        \end{itemize}
    }

    \frame{\frametitle{Global Illumination}\framesubtitle{Radiosity}
        \[B_i = E_i + \rho_i \sum_{j=1}^n F_{ij} B_j\]
        \begin{description}
            \item[+] Viewport independent (offline processing)
            \item[-] No specular or glossy reflections
            \item[+-] Simplified rendering equation
        \end{description}
        \begin{figure}
        \centering
        \includegraphics[width=0.9\textwidth]{share/radiosity.png}
        \end{figure}
    }

    \frame{\frametitle{Global Illumination}\framesubtitle{Whitted Raytracing}
        \begin{description}
            \item[+] Specular and glossy surfaces
            \item[-] Viewport dependent
            \item[-] Only point lights
            \item[-] Local diffuse lighting model
        \end{description}
        \begin{figure}
        \centering
        \includegraphics[width=0.7\textwidth]{share/whitted.png}
        \end{figure}
    }
    
    \frame{\frametitle{Global Illumination}\framesubtitle{Path tracing}
        \begin{description}
            \item[+] Photo realistic
            \item[-] Super slow (otherwise noisy)
            \item[-] Expensive caustics
            \item[-] Volumetric light scattering
        \end{description}
        \begin{figure}
        \centering
        \includegraphics[width=0.7\textwidth]{share/pathtracing.png}
        \end{figure}
    }

    \frame{\frametitle{Photon Mapping}\framesubtitle{Overview}
        An extension to path tracing which uses \emph{photon maps} for faster convergence.
        \begin{itemize}
            \item Easier caustics
            \item Volumetric material properties
            \item Memory requirement
            \item Photon maps viewport independent
            \item 2-pass rendering
        \end{itemize}
    }
    
    \frame{\frametitle{Photon Mapping}\framesubtitle{1st pass}
        
    }
    
    \frame{\frametitle{Photon Mapping}\framesubtitle{2nd pass}
    
    }

    \frame{\frametitle{Discussion}
    
    }

    \frame{\frametitle{Future Studies}

    }

    \frame{\frametitle{Any Questions?}}
    \frame{\frametitle{Bibliography}
        \nocite{*}
        \bibliographystyle{alpha}
        \bibliography{slides}
    }
\end{document}
