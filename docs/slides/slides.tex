\documentclass{beamer}
\usepackage[utf8]{inputenc}
\usepackage{algorithmic}
\usepackage{algorithm}
\usepackage{amsfonts}
\usepackage{amssymb}
\usepackage{courier}
\usepackage{graphicx}
\usepackage{listings}
\usepackage{mathtools}
\usepackage[font={small}, labelfont={color=black}]{caption}
\usetheme{Rochester}

\usefonttheme[onlymath]{serif}
\beamertemplatenavigationsymbolsempty
\title{\LARGE{Global Illumination Using Photon Maps}\\
       \large{\emph{A summary of the paper by Henrik W. Jensen~\cite{jensen1996global}}}}
\author{\vspace{2ex}\\\textbf{Martin Estegren} \;\;\,\,
        \texttt{<mares480@student.liu.se>} \\
        \textbf{Erik S. V. Jansson}\;
        \texttt{<erija578@student.liu.se>} \\~\\
        {Advanced Global Illumination and Rendering}\\
        {at ITN Linköping University (LiTH), Sweden}}

\lstset{
    escapeinside={<@}{@>},
    basicstyle=\tiny\ttfamily,
    breakatwhitespace = false,
    breaklines = true,
    captionpos = b,
    keepspaces = true,
    language = C++,
    showspaces = false,
    showstringspaces = false,
    frame = tb,
    aboveskip = 10pt,
    belowskip = 10pt,
    numbers = left,
    numbersep = 3pt
}

\setbeamerfont{bibliography item}{size=\tiny}
\setbeamerfont{bibliography entry author}{size=\tiny}
\setbeamerfont{bibliography entry title}{size=\tiny}
\setbeamerfont{bibliography entry location}{size=\tiny}
\setbeamerfont{bibliography entry note}{size=\tiny}
\setbeamertemplate{footline}[frame number]
\begin{document}
    \frame{\titlepage}

    \frame{\frametitle{Outline}
        \begin{itemize}
            \item Motivation
            \item Global Illumination
                \begin{itemize}
                    \item Rendering Equation
                    \item Radiosity
                    \item Whitted Raytracing
                    \item Path Tracing
                \end{itemize}
            \item Photon Mapping
                \begin{itemize}
                    \item Photon Tracing
                    \item Radiance Estimate
                    \item Photon\; Collection
                \end{itemize}
            \item Summary
            \item Further Studies
        \end{itemize}
    }

    \frame{\frametitle{Motivation}
        \begin{columns}
        \column{0.64\textwidth}
            \begin{itemize}
                \item We want a global illumination scheme that achieves photorealism as \(N \rightarrow \infty\)
                \item It should also preferably be noise-free!
                \item Should be a physically sound model
                \item Preferably also fast to compute!
            \end{itemize}
        \column{0.36\textwidth} \begin{center}
            \includegraphics[height=0.8\textheight]{share/dragon.jpg}
        \end{center} \end{columns}
    }

    \frame{\frametitle{Global Illumination}\framesubtitle{Rendering Equation~\cite{kajiya1986rendering}}
        \[L_o(\vec{x}, \hat{\omega}_o) = L_e(\vec{x}, \hat{\omega}_o) +
                                         \overbrace{\int_\Omega L_i(\vec{x}, \hat{\omega}_i)
                                         f_r(\vec{x}, \hat{\omega}_i, \hat{\omega}_o)
                                         (\hat{n}_{x} \cdot \hat{\omega}_i) \, d\hat{\omega}_i}
                                         ^{\text{\emph{reflected radiance} \(L_r\) from } \Omega \text{ to } \vec{x} \text{ toward } \hat{\omega}_o}\]
        \begin{itemize}
            \item \(L_o(\vec{x}, \hat{\omega}_o)\): total \emph{outgoing radiance} at point \(\vec{x}\) towards a \(\hat{\omega}_o\).
            \item \(L_e(\vec{x}, \hat{\omega}_o)\): \emph{emitted radiance} contribution from \(\vec{x}\) toward \(\hat{\omega}_o\).
            \item \(\Omega\): hemisphere around the point \(\vec{x}\) with normal \(\hat{n}_x\) of \(d \hat{\omega}_i\)'s.
            \item \(L_i(\vec{x}, \hat{\omega}_i)\):\ \emph{incoming radiance} contributions fr.\ \(\hat{\omega}_i\) towards \(\vec{x}\).
            \item \(f_r(\vec{x}, \hat{\omega}_i, \hat{\omega}_o)\): surface \emph{reflectance properties} at \(\vec{x}\), an BRDF.
        \end{itemize}
    }

    \frame{\frametitle{Global Illumination}\framesubtitle{Radiosity~\cite{goral1984modeling}}
        \begin{columns}
        \column{0.64\textwidth}
            Assumes all surfaces are Lambertian reflectors: \(f_r = \rho / \pi\) of discrete size.
            \[B_i = E_i + \rho_i \sum_{j=1}^n F_{ij} B_j\]
            \begin{itemize}
                \item[$+$] Noise free,\, iterative progression
                \item[$+$] Viewport independent (baking!)
                \item[$+$] Accurate for Lambertian surface
                \item[$-$] No specular or glossy reflections
                \item[$-$] Complexity scales \(\propto\) to triangles
            \end{itemize}
        \column{0.36\textwidth} \begin{center}
            \includegraphics[height=0.8\textheight]{share/radiosity.png}
        \end{center} \end{columns}
    }

    \frame{\frametitle{Global Illumination}\framesubtitle{Whitted Raytracing~\cite{whitted1979improved}}
        \begin{columns}
        \column{0.64\textwidth}
            \begin{itemize}
                \item[$\sim$] Viewport dependent (no baking)
                \item[$-$] No soft shadows, $\rightarrow$ point lights
                \item[$+$] Enables fully specular reflections
                \item[$-$] Uses local model for diffuse surf.
                \item[$+$] OK to parallelize and implement
            \end{itemize}
        \column{0.36\textwidth} \begin{center}
            \includegraphics[height=0.8\textheight]{share/whitted.png}
        \end{center} \end{columns}
    }

    \frame{\frametitle{Global Illumination}\framesubtitle{Path Tracing}
        \begin{columns}
        \column{0.64\textwidth}
            \begin{itemize}
                \item[$+$] Models almost all light transport
                \item[$+$] Embarrassingly parallel algorithm
                \item[$-$] High-freq.\, noise if undersampled
                \item[$-$] Doesn't consider vol.\, interaction
                \item[$-$] Unfeasible for ``detailed caustics''
            \end{itemize}
        \column{0.36\textwidth} \begin{center}
            \includegraphics[height=0.8\textheight]{share/pathtracing.png}
        \end{center} \end{columns}
    }

    \frame{\frametitle{Photon Mapping}
        \begin{columns}
        \column{0.64\textwidth}
            Efficient ray tracing extension that features faster \emph{caustics}, and, \emph{sub-surface scattering}. Trick:\, send \emph{``photons''} from a light source.

            \vspace{1.5em}

            Algorithm is divided into two major passes:
            \begin{itemize}
                \item Photon tracing: emit photons carrying flux \(d\Phi\) from light sources towards our scene and stores hits on \emph{photon maps}.
                \item Photon collection: estimate irradiance at \(\vec{x}\) by integrating over \emph{photon maps}. Either \emph{accurate}, or \emph{approximate} mode.
            \end{itemize}
        \column{0.36\textwidth} \begin{center}
            \includegraphics[height=0.8\textheight]{share/photonmapping.jpg}
        \end{center} \end{columns}
    }

    \frame{\frametitle{Photon Tracing}

        Photons are emitted from light source and traced similar to path tracing and terminated on diffuse surfaces.

        \textbf{Caustics} \(\approx L_{i,c}(\vec{x},\hat{\omega}_i)\)
        \begin{itemize}
            \item Photons are only emitted towards specular objects
	    \item Used to directly visualize caustics
        \end{itemize}

        \textbf{Global} \(\approx L_{i,d}(\vec{x},\hat{\omega}_i)\)
        \begin{itemize}
            \item Photons are emitted uniformly over the hemisphere
            \item Only used as a data structure
        \end{itemize}
        
    }

    \frame{\frametitle{Photon Tracing} \framesubtitle{Photon-Surface Interaction}
        
	Surface interaction is determined by russian roulette. 
        
        \vspace{1.5em}
        
        Assume a uniform real distribution \(\xi \in [0,1]\) and the surface coefficients \(r\) and \(t\):
        
        \begin{itemize}
            \item \( \xi \in [0, r] \) - Reflection of photon 
            \item \( \xi \in [r, r + t] \) - Transmission of photon 
            \item \( \xi \in [r + t, 1] \) - Absorption of photon
        \end{itemize}
        
    }

    \frame{\frametitle{Radiance Estimate}
        
    The reflected radiance from a point \(\vec{x}\) can be estimated as:

    \begin{align*}
        &L_r(\vec{x}, \hat\omega_o) \approx \sum_{p = 1}^{n} f_r(\vec{x},\hat\omega_{i,p}, \hat\omega_o)\frac{\Delta\Phi_p(\vec{x},\hat\omega_{i,p})}{\pi r^2}
    \end{align*}

    Breaking down the estimate:
    \begin{enumerate}
        \item Locate the \(n\) nearest photons \(p\) with \emph{flux} \(\Delta\Phi_p\) around \(\vec{x}\)
        \item Approximate the area containing the photons as \(\pi r^2\) (circle)
        \item For each photon: Multiply the \textit{BRDF} \(f_r(\vec{x}, \hat\omega_{i,p}, \hat\omega_o)\) with the \textit{flux} divided by the approximate containment area
    \end{enumerate}      
    \vspace{1.0em}                
    
    Possible optimization: By using a fixed bounding sphere around \(\vec{x}\) instead of the \(n\) nearest photons.  
    }

    \frame{\frametitle{Photon Collection}
        After we have our photon maps of the scene, we use distribution raytracing to compute each pixel's average radiance by sampling estimates from the scene: \(L_o(\vec{x}, \hat\omega_o) = L_e(\vec{x}, \hat\omega_o) + L_r(\vec{x}, \hat\omega_o)\). In photon mapping we split \(L_r(\vec{x}, \hat\omega_o)\) into 4 radiance contributions:
            \begin{align*}
                &L_r(\vec{x}, \hat\omega_o) = \int_\Omega L_i(\vec{x}, \hat{\omega}_i)
                                             f_r(\vec{x}, \hat{\omega}_i, \hat{\omega}_o)
                                             (\hat{n}_{x} \cdot \hat{\omega}_i) \, d\hat{\omega}_i = \\
                                           &L_{r,l}(\vec{x}, \hat\omega_o) + L_{r,s}(\vec{x}, \hat\omega_o) + L_{r,c}(\vec{x}, \hat\omega_o) + L_{r,d}(\vec{x}, \hat\omega_o)
            \end{align*}
        Each of these are evaluated either \emph{accurately} or \emph{approximatively}\ :
        \begin{itemize}
            \item Accurate evaluation: when \(\vec{x}\) is seen by an eye directly/close
            \item Approximate evaluation:\ a importance was reflected diffusely
        \end{itemize}
    }

    \frame{\frametitle{Photon Collection}\framesubtitle{Direct Illumination}
        Represents the reflected radiance at \(\vec{x}\) towards \(\hat\omega_o\) which originated directly from \(\hat\omega_i\), the sources of light found around a hemisphere \(\Omega\).
        \vspace{1em}
        \[L_{r,l}(\vec{x}, \hat\omega_o) = \int_\Omega L_{i,l}(\vec{x}, \hat{\omega}_i)
                                           f_{r,d}(\vec{x}, \hat{\omega}_i, \hat{\omega}_o)
                                           (\hat{n}_{x} \cdot \hat{\omega}_i) \, d\hat{\omega}_i\]
        \begin{itemize}
            \item Accurate evaluation: we would have to launch \emph{shadow rays} in the same way as Monte Carlo raytracing.\ Sample if area light.
            \item Approximate evaluation: the radiance estimate obtained from our \emph{global photon map}, sum estimated photon flux around \(\vec{x}\).
        \end{itemize}
    }

    \frame{\frametitle{Photon Collection}\framesubtitle{Specular and Glossy Reflections}
        Specular and glossy reflections are handled using the sum over the incoming caustic and diffuse radiance with the specular BRDF \(f_{r,s}\). No approximated variant is needed, we only have a single direction.
        \vspace{-0.2em}
        \begin{align*}
            L_{r,s}(\vec{x}, \hat\omega_o) &= \int_\Omega L_{i,c}(\vec{x}, \hat{\omega}_i)
                                             f_{r,s}(\vec{x}, \hat{\omega}_i, \hat{\omega}_o)
                                             (\hat{n}_{x} \cdot \hat{\omega}_i) \, d\hat{\omega}_i \\
                                           &+ \int_\Omega L_{i,d}(\vec{x}, \hat{\omega}_i)
                                             f_{r,s}(\vec{x}, \hat{\omega}_i, \hat{\omega}_o)
                                             (\hat{n}_{x} \cdot \hat{\omega}_i) \, d\hat{\omega}_i
        \end{align*}
        \begin{itemize}
            \item Accurate evaluation: we use Monte Carlo raytracing and \(f_{r,s}\) distributed importance sampling for reflection \& transmission.
        \end{itemize}
    }

    \frame{\frametitle{Photon Collection}\framesubtitle{Caustics}
        Contributions from caustics are never evaluated using Monte Carlo raytracing, since it's very expensive. We use photon maps instead. \(L_{i,c}(\vec{x}, \hat\omega_o)\) is indirect light via specular reflection, or, transmission.
        \vspace{0.8em}
        \[L_{r,c}(\vec{x}, \hat\omega_o) = \int_\Omega L_{i,c}(\vec{x}, \hat{\omega}_i)
                                           f_{r,d}(\vec{x}, \hat{\omega}_i, \hat{\omega}_o)
                                           (\hat{n}_{x} \cdot \hat{\omega}_i) \, d\hat{\omega}_i\]
        \begin{itemize}
            \item Accurate evaluation: we use radiance estimate of the \emph{caustics photon map}. This is why it needs to be a high-resolution map.
            \item Approximate evaluation: again, we use the global photon map.
        \end{itemize}
    }

    \frame{\frametitle{Photon Collection}\framesubtitle{Indirect Diffuse Reflections}
        Finally, for contributions arriving at \(\vec{x}\) which have bounced around diffusely at least once we integrate over \(L_{i,d}(\vec{x}, \hat\omega_i)\). This gives the visual effects commonly known as ``color bleeding'', as in radiosity.
        \vspace{0.8em}
        \[L_{r,d}(\vec{x}, \hat\omega_o) = \int_\Omega L_{i,d}(\vec{x}, \hat{\omega}_i)
                                           f_{r,d}(\vec{x}, \hat{\omega}_i, \hat{\omega}_o)
                                           (\hat{n}_{x} \cdot \hat{\omega}_i) \, d\hat{\omega}_i\]
        \begin{itemize}
            \item Accurate evaluation: we use Monte Carlo raytracing again for this, and an optimzed sampling distribution according to \(f_{r,d}\).
            \item Approximate evaluation: will usually contribute quite little to the end-results, so we again use the global radiance estimate.
        \end{itemize}
    }

    \frame{\frametitle{Photon Mapping} \framesubtitle{Summary}
        \begin{itemize}
            \item We emit photons from light sources and store them in a data structure called a photon map
            \item With the help of these photon maps we can estimate the radiance falling on a surface point
            \item Finally, we path trace through the scene more effectively by using our photon maps
        \end{itemize}
    }

    \frame{\frametitle{Summary}
        \begin{columns}
        \column{0.64\textwidth}
            In this presentation we have discussed:
            \begin{itemize}
                \item Why we should care about global illumination at all
                \item Existing global illumination models
                \item How photon mapping improves on path tracing
            \end{itemize}
        \column{0.36\textwidth} \begin{center}
            \includegraphics[height=0.8\textheight]{share/otherpathtracing.png}
        \end{center} \end{columns}
    }

    \frame{\frametitle{Further Studies}
        \begin{itemize}
            \item Improved radiance estimate: we add a filter based on the distance which re-weight the contributions from photons.
            \item Volume photon map: models interactions in participating media. We need a new radiance estimate and a ``BSDF''.
            \item Photon splatting: instead of storing a photon individually,\\we accumulate the radiance in a texture\, with all the hits.
        \end{itemize}
    }

    \frame{\frametitle{Any Questions?}}
    \frame{\frametitle{Bibliography}
        \nocite{*}
        \bibliographystyle{alpha}
        \bibliography{slides}
    }
\end{document}
