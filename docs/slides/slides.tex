\documentclass{beamer}
\usepackage[utf8]{inputenc}
\usepackage{algorithmic}
\usepackage{algorithm}
\usepackage{amsfonts}
\usepackage{amssymb}
\usepackage{courier}
\usepackage{graphicx}
\usepackage{listings}
\usepackage{mathtools}
\usepackage[font={small}, labelfont={color=black}]{caption}
\usetheme{Rochester}

\usefonttheme[onlymath]{serif}
\beamertemplatenavigationsymbolsempty
\title{\LARGE{Global Illumination Using Photon Maps}\\
       \large{\emph{A summary of the paper by Henrik W. Jensen~\cite{jensen1996global}}}}
\author{\vspace{2ex}\\\textbf{Martin Estegren} \;\;\,\,
        \texttt{<mares480@student.liu.se>} \\
        \textbf{Erik S. V. Jansson}\;
        \texttt{<erija578@student.liu.se>} \\~\\
        {Advanced Global Illumination and Rendering}\\
        {at ITN Linköping University (LiTH), Sweden}}

\lstset{
    escapeinside={<@}{@>},
    basicstyle=\tiny\ttfamily,
    breakatwhitespace = false,
    breaklines = true,
    captionpos = b,
    keepspaces = true,
    language = C++,
    showspaces = false,
    showstringspaces = false,
    frame = tb,
    aboveskip = 10pt,
    belowskip = 10pt,
    numbers = left,
    numbersep = 3pt
}

\setbeamertemplate{footline}[frame number]
\begin{document}
    \frame{\titlepage}

    \frame{\frametitle{Outline}
        \begin{itemize}
            \item Motivation
            \item Global Illumination
                \begin{itemize}
                    \item Rendering Equation
                    \item Radiosity
                    \item Whitted Raytracing
                    \item Path Tracing
                \end{itemize}
            \item Photon Mapping
                \begin{itemize}
                    \item Photon Tracing
                    \item Photon Collection
                \end{itemize}
            \item Summary
            \item Further Studies
        \end{itemize}
    }

    \frame{\frametitle{Motivation}
    }

    \frame{\frametitle{Global Illumination}\framesubtitle{Rendering Equation}
        \[L_o(\vec{x}, \hat{\omega}_o) = L_e(\vec{x}, \hat{\omega}_o) +
                                         \overbrace{\int_\Omega L_i(\vec{x}, \hat{\omega}_i)
                                         f_r(\vec{x}, \hat{\omega}_i, \hat{\omega}_o)
                                         (\hat{n}_{x} \cdot \hat{\omega}_i) \, d\hat{\omega}_i}
                                         ^{\text{\emph{reflected radiances} from } \Omega \text{ into } \vec{x} \text{ toward } \hat{\omega}_o}\]
        \begin{itemize}
            \item \(L_o(\vec{x}, \hat{\omega}_o)\): total \emph{outgoing radiance} at point \(\vec{x}\) towards a \(\hat{\omega}_o\).
            \item \(L_e(\vec{x}, \hat{\omega}_o)\): \emph{emitted radiance} contribution from \(\vec{x}\) toward \(\hat{\omega}_o\).
            \item \(\Omega\): hemisphere around the point \(\vec{x}\) with normal \(\hat{n}_x\) of \(d \hat{\omega}_i\)'s.
            \item \(L_i(\vec{x}, \hat{\omega}_i)\):\ \emph{incoming radiance} contributions fr.\ \(\hat{\omega}_i\) towards \(\vec{x}\).
            \item \(f_r(\vec{x}, \hat{\omega}_i, \hat{\omega}_o)\): surface \emph{reflectance properties} at \(\vec{x}\), an BRDF.
        \end{itemize}
    }

    \frame{\frametitle{Global Illumination}\framesubtitle{Radiosity}
        \begin{columns}
        \column{0.64\textwidth}
            \[B_i = E_i + \rho_i \sum_{j=1}^n F_{ij} B_j\]
            \begin{itemize}
                \item[$+$] Noise free, iterative
                \item[$+$] Viewport independent (baking!)
                \item[$+$] Accurate for Lambertian surface
                \item[$-$] No specular or glossy reflections
                \item[$-$] Complexity scales with triangles
            \end{itemize}
        \column{0.36\textwidth} \begin{center}
            \includegraphics[height=0.8\textheight]{share/radiosity.png}
        \end{center} \end{columns}
    }

    \frame{\frametitle{Global Illumination}\framesubtitle{Whitted Raytracing}
        \begin{columns}
        \column{0.64\textwidth}
            \begin{itemize}
                \item[$-$] Viewport dependent
                \item[$-$] No area light sources
                \item[$+$] Specular and glossy surfaces
                \item[$-$] Local model for diffuse surfaces
                \item[$+$] Easy to parallelize and to setup
            \end{itemize}
        \column{0.36\textwidth} \begin{center}
            \includegraphics[height=0.8\textheight]{share/whitted.png}
        \end{center} \end{columns}
    }

    \frame{\frametitle{Global Illumination}\framesubtitle{Path Tracing}
        \begin{columns}
        \column{0.64\textwidth}
            \begin{itemize}
                \item[$+$] Photo realistic
                \item[$-$] Super slow (otherwise noisy)
                \item[$-$] Expensive caustics
                \item[$-$] Volumetric light scattering
            \end{itemize}
        \column{0.36\textwidth} \begin{center}
            \includegraphics[height=0.8\textheight]{share/pathtracing.png}
        \end{center} \end{columns}
    }

    \frame{\frametitle{Photon Mapping}
        \begin{columns}
        \column{0.64\textwidth}
            Efficient ray tracing extension that features \emph{faster caustics}, and, \emph{sub-surface scattering}.

            \vspace{1.5em}

            Algorithm is divided into two major passes:
            \begin{itemize}
                \item Photon tracing: emit photons carrying flux \(d\Phi\) from light sources towards our scene and stores hits on \emph{photon maps}.
                \item Photon collection: estimate irradiance at \(\vec{x}\) by integrating over \emph{photon maps}.
            \end{itemize}
        \column{0.36\textwidth} \begin{center}
            \includegraphics[height=0.8\textheight]{share/photonmapping.jpg}
        \end{center} \end{columns}
    }

    \frame{\frametitle{Photon Tracing}
    }

    \frame{\frametitle{Photon Collection}
    }

    \frame{\frametitle{Summary}
    }

    \frame{\frametitle{Further Studies}
    }

    \frame{\frametitle{Any Questions?}}
    \frame{\frametitle{Bibliography}
        \nocite{*}
        \bibliographystyle{alpha}
        \bibliography{slides}
    }
\end{document}
