\documentclass{beamer}
\usepackage[utf8]{inputenc}
\usepackage{algorithmic}
\usepackage{algorithm}
\usepackage{amsfonts}
\usepackage{amssymb}
\usepackage{courier}
\usepackage{graphicx}
\usepackage{listings}
\usepackage{mathtools}
\usepackage[font={small}, labelfont={color=black}]{caption}
\usetheme{Rochester}

\usefonttheme[onlymath]{serif}
\beamertemplatenavigationsymbolsempty
\title{\LARGE{Global Illumination Using Photon Maps}\\
       \large{\emph{A summary of the paper by Henrik W. Jensen~\cite{jensen1996global}}}}
\author{\vspace{2ex}\\\textbf{Martin Estegren} \;\;\,\,
        \texttt{<mares480@student.liu.se>} \\
        \textbf{Erik S. V. Jansson}\;
        \texttt{<erija578@student.liu.se>} \\~\\
        {Advanced Global Illumination and Rendering}\\
        {at ITN Linköping University (LiTH), Sweden}}

\lstset{
    escapeinside={<@}{@>},
    basicstyle=\tiny\ttfamily,
    breakatwhitespace = false,
    breaklines = true,
    captionpos = b,
    keepspaces = true,
    language = C++,
    showspaces = false,
    showstringspaces = false,
    frame = tb,
    aboveskip = 10pt,
    belowskip = 10pt,
    numbers = left,
    numbersep = 3pt
}

\setbeamertemplate{footline}[frame number]
\begin{document}
    \frame{\titlepage}

    \frame{\frametitle{Outline}
        \begin{itemize}
            \item Motivation
            \item Global Illumination
                \begin{itemize}
                    \item Rendering Equation
                    \item Radiosity
                    \item Whitted Raytracing
                    \item Path Tracing
                \end{itemize}
            \item Photon Mapping
                \begin{itemize}
                    \item Photon Tracing
                    \item Radiance Estimate
                    \item Photon\; Collection
                \end{itemize}
            \item Summary
            \item Further Studies
        \end{itemize}
    }

    \frame{\frametitle{Motivation}
    }

    \frame{\frametitle{Global Illumination}\framesubtitle{Rendering Equation}
        \[L_o(\vec{x}, \hat{\omega}_o) = L_e(\vec{x}, \hat{\omega}_o) +
                                         \overbrace{\int_\Omega L_i(\vec{x}, \hat{\omega}_i)
                                         f_r(\vec{x}, \hat{\omega}_i, \hat{\omega}_o)
                                         (\hat{n}_{x} \cdot \hat{\omega}_i) \, d\hat{\omega}_i}
                                         ^{\text{\emph{reflected radiance} \(L_r\) from } \Omega \text{ to } \vec{x} \text{ toward } \hat{\omega}_o}\]
        \begin{itemize}
            \item \(L_o(\vec{x}, \hat{\omega}_o)\): total \emph{outgoing radiance} at point \(\vec{x}\) towards a \(\hat{\omega}_o\).
            \item \(L_e(\vec{x}, \hat{\omega}_o)\): \emph{emitted radiance} contribution from \(\vec{x}\) toward \(\hat{\omega}_o\).
            \item \(\Omega\): hemisphere around the point \(\vec{x}\) with normal \(\hat{n}_x\) of \(d \hat{\omega}_i\)'s.
            \item \(L_i(\vec{x}, \hat{\omega}_i)\):\ \emph{incoming radiance} contributions fr.\ \(\hat{\omega}_i\) towards \(\vec{x}\).
            \item \(f_r(\vec{x}, \hat{\omega}_i, \hat{\omega}_o)\): surface \emph{reflectance properties} at \(\vec{x}\), an BRDF.
        \end{itemize}
    }

    \frame{\frametitle{Global Illumination}\framesubtitle{Radiosity}
        \begin{columns}
        \column{0.64\textwidth}
            Assumes all surfaces are Lambertian reflectors: \(f_r = \rho / \pi\) of discrete size.
            \[B_i = E_i + \rho_i \sum_{j=1}^n F_{ij} B_j\]
            \begin{itemize}
                \item[$+$] Noise free,\, iterative progression
                \item[$+$] Viewport independent (baking!)
                \item[$+$] Accurate for Lambertian surface
                \item[$-$] No specular or glossy reflections
                \item[$-$] Complexity scales \(\propto\) to triangles
            \end{itemize}
        \column{0.36\textwidth} \begin{center}
            \includegraphics[height=0.8\textheight]{share/radiosity.png}
        \end{center} \end{columns}
    }

    \frame{\frametitle{Global Illumination}\framesubtitle{Whitted Raytracing}
        \begin{columns}
        \column{0.64\textwidth}
            \begin{itemize}
                \item[$\sim$] Viewport dependent (no baking)
                \item[$-$] No soft shadows, $\rightarrow$ point lights
                \item[$+$] Enables fully specular reflections
                \item[$-$] Uses local model for diffuse surf.
                \item[$+$] OK to parallelize and implement
            \end{itemize}
        \column{0.36\textwidth} \begin{center}
            \includegraphics[height=0.8\textheight]{share/whitted.png}
        \end{center} \end{columns}
    }

    \frame{\frametitle{Global Illumination}\framesubtitle{Path Tracing}
        \begin{columns}
        \column{0.64\textwidth}
            \begin{itemize}
                \item[$+$] Models almost all light transport
                \item[$+$] Embarrassingly parallel algorithm
                \item[$-$] High-freq.\, noise if undersampled
                \item[$-$] Doesn't consider vol.\, interaction
                \item[$-$] Unfeasible for ``detailed caustics''
            \end{itemize}
        \column{0.36\textwidth} \begin{center}
            \includegraphics[height=0.8\textheight]{share/pathtracing.png}
        \end{center} \end{columns}
    }

    \frame{\frametitle{Photon Mapping}
        \begin{columns}
        \column{0.64\textwidth}
            Efficient ray tracing extension that features faster \emph{caustics}, and, \emph{sub-surface scattering}. Trick:\, send \emph{``photons''} from a light source.

            \vspace{1.5em}

            Algorithm is divided into two major passes:
            \begin{itemize}
                \item Photon tracing: emit photons carrying flux \(d\Phi\) from light sources towards our scene and stores hits on \emph{photon maps}.
                \item Photon collection: estimate irradiance at \(\vec{x}\) by integrating over \emph{photon maps}. Either \emph{accurate}, or \emph{approximate} mode.
            \end{itemize}
        \column{0.36\textwidth} \begin{center}
            \includegraphics[height=0.8\textheight]{share/photonmapping.jpg}
        \end{center} \end{columns}
    }

    \frame{\frametitle{Photon Tracing}
    \begin{columns}
        \column{0.8\textwidth}

        Photons are emitted from light source and traced similar to path tracing.

        \textbf{Caustics}
        \begin{itemize}
            \item Photons are only emitted towards specular objects
            \item Stored whenever they hit diffuse surfaces
        \end{itemize}

        \textbf{Global}
        \begin{itemize}
            \item Photons are emitted uniformly
            \item Can be less precise 
            \item First hit stored as normal photon
            \item Secondaries stored as shadow photons  
        \end{itemize}
        
        \column{0.2\textwidth} \begin{center}
        \end{center} \end{columns}
    }

    \frame{\frametitle{Photon Tracing} \framesubtitle{photon-surface interaction}
    \begin{columns}
        \column{0.8\textwidth}
        Surface interaction is determined by russian roulette. 
        
        \vspace{1.5em}
        
        Assume a uniform real distribution \(\xi \in [0,1]\) and the surface coefficients \(d\) and \(s\):
        
        \begin{itemize}
            \item \( \xi \in [0, d] \) - Reflection of photon 
            \item \( \xi \in [d, s + d] \) - Transmission of photon 
            \item \( \xi \in [s + d, 1] \) - Absorption of photon
        \end{itemize}
        
        \column{0.2\textwidth} \begin{center}
        \end{center} \end{columns}
    }

    \frame{\frametitle{Radiance Estimate}
        
    The outgoing radiance from a point \(\vec{x}\) can be estimated as: 

    \begin{align*}
        &L_r \approx \sum_{p = 1}^{n} f_r(\vec{x},\Psi_r,\Psi_{i,p})\frac{d\Phi_p(\vec{x},\Psi_{i,p})}{\pi r^2}
    \end{align*}

    Breaking down the estimate:
    \begin{enumerate}
       \item Locate the \(n\) nearest photons \(\Phi_p\) to \(\vec{x}\)
       \item Approximate the area containing the photons as \(\pi r^2\)
       \item For each photon: Multiply the \textit{BRDF} \(f_r(\vec{x},\Psi_r,\Psi_{i,p})\) with the \textit{flux} divided by the approximate containment area
    \end{enumerate}      
    \vspace{1.0em}                
    
    Possible optimization: By using a fixed bounding sphere around \(\vec{x}\) instead of the \(n\) nearest photons.  
    }

    \frame{\frametitle{Photon Collection}
            \begin{align*}
                &L_r(\vec{x}, \hat\omega_o) = \int_\Omega L_i(\vec{x}, \hat{\omega}_i)
                                             f_r(\vec{x}, \hat{\omega}_i, \hat{\omega}_o)
                                             (\hat{n}_{x} \cdot \hat{\omega}_i) \, d\hat{\omega}_i = \\
                                           &L_{r,l}(\vec{x}, \hat\omega_o) + L_{r,s}(\vec{x}, \hat\omega_o) + L_{r,c}(\vec{x}, \hat\omega_o) + L_{r,d}(\vec{x}, \hat\omega_o)
            \end{align*}
    }

    \frame{\frametitle{Photon Collection}\framesubtitle{Direct Illumination}
        \[L_{r,l}(\vec{x}, \hat\omega_o) = \int_\Omega L_{i,l}(\vec{x}, \hat{\omega}_i)
                                           f_{r,l}(\vec{x}, \hat{\omega}_i, \hat{\omega}_o)
                                           (\hat{n}_{x} \cdot \hat{\omega}_i) \, d\hat{\omega}_i\]
    }

    \frame{\frametitle{Photon Collection}\framesubtitle{Specular and Glossy Reflections}
        \begin{align*}
            L_{r,s}(\vec{x}, \hat\omega_o) &= \int_\Omega L_{i,c}(\vec{x}, \hat{\omega}_i)
                                             f_{r,s}(\vec{x}, \hat{\omega}_i, \hat{\omega}_o)
                                             (\hat{n}_{x} \cdot \hat{\omega}_i) \, d\hat{\omega}_i \\
                                           &+ \int_\Omega L_{i,d}(\vec{x}, \hat{\omega}_i)
                                             f_{r,s}(\vec{x}, \hat{\omega}_i, \hat{\omega}_o)
                                             (\hat{n}_{x} \cdot \hat{\omega}_i) \, d\hat{\omega}_i
        \end{align*}
    }

    \frame{\frametitle{Photon Collection}\framesubtitle{Caustics}
        \[L_{r,c}(\vec{x}, \hat\omega_o) = \int_\Omega L_{i,c}(\vec{x}, \hat{\omega}_i)
                                           f_{r,d}(\vec{x}, \hat{\omega}_i, \hat{\omega}_o)
                                           (\hat{n}_{x} \cdot \hat{\omega}_i) \, d\hat{\omega}_i\]
    }

    \frame{\frametitle{Photon Collection}\framesubtitle{Indirect Diffuse Reflections}
        \[L_{r,d}(\vec{x}, \hat\omega_o) = \int_\Omega L_{i,d}(\vec{x}, \hat{\omega}_i)
                                           f_{r,d}(\vec{x}, \hat{\omega}_i, \hat{\omega}_o)
                                           (\hat{n}_{x} \cdot \hat{\omega}_i) \, d\hat{\omega}_i\]
    }

    \frame{\frametitle{Summary}
    }

    \frame{\frametitle{Further Studies}
        \begin{itemize}
            \item Improved radiance estimate:
            \item Volume photon map:
            \item Photon relaxation:
            \item Photon splatting:
            \item Adaptive photon tracing:
            \item Progressive photon mapping:
        \end{itemize}
    }

    \frame{\frametitle{Any Questions?}}
    \frame{\frametitle{Bibliography}
        \nocite{*}
        \bibliographystyle{alpha}
        \bibliography{slides}
    }
\end{document}
